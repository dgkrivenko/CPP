\documentclass[pdf, unicode, 12pt, a4paper,oneside,fleqn]{article}
\usepackage[utf8]{inputenc}
\usepackage[T2B]{fontenc}
\usepackage[english,russian]{babel}

\frenchspacing

\usepackage{amsmath}
\usepackage{amssymb}
\usepackage{hyperref}
\usepackage{longtable}
\usepackage[table]{xcolor}  
\usepackage{array}
\usepackage{color}
\usepackage{xcolor}

\usepackage{hyperref}


\newcommand{\MYhref}[3][blue]{\href{#2}{\color{#1}{#3}}}%

\usepackage{listings}
\usepackage{alltt}
\usepackage{csquotes}
\DeclareQuoteStyle{russian}
	{\guillemotleft}{\guillemotright}[0.025em]
	{\quotedblbase}{\textquotedblleft}
\ExecuteQuoteOptions{style=russian}

\usepackage{graphicx}

\usepackage{listings}
\lstset{tabsize=2,
	breaklines,
	columns=fullflexible,
	flexiblecolumns,
	numbers=left,
	numberstyle={\footnotesize},
	extendedchars,
	inputencoding=utf8}

\usepackage{longtable}

\def\@xobeysp{ }
\def\verbatim@processline{\hspace{1.2cm}\raggedright\the\verbatim@line\par}

\oddsidemargin=-0.4mm
\textwidth=160mm
\topmargin=4.6mm
\textheight=210mm

\parindent=0pt
\parskip=3pt

\definecolor{lightgray}{gray}{0.9}


\renewcommand{\thesubsection}{\arabic{subsection}}

\lstdefinestyle{customc}{
  belowcaptionskip=1\baselineskip,
  breaklines=true,
  frame=L,
  xleftmargin=\parindent,
  language=C,
  showstringspaces=false,
  basicstyle=\footnotesize\ttfamily,
  keywordstyle=\bfseries\color{green!40!black},
  commentstyle=\itshape\color{gray},
  identifierstyle=\color{black},
  stringstyle=\color{blue},
}

\lstdefinestyle{customasm}{
  belowcaptionskip=1\baselineskip,
  frame=L,
  xleftmargin=\parindent,
  language=[x86masm]Assembler,
  basicstyle=\footnotesize\ttfamily,
  commentstyle=\itshape\color{purple!40!black},
}

\lstset{escapechar=@,style=customc}


\newcommand{\CWPHeader}[1]{\addtocounter{section}{-1}\section{#1}}

% Заголовок курсовой работы
% Единственный аргумент --- ее тема
\newcommand{\CWHeader}[1]{\section*{#1}}

\newcommand{\CWProblem}[1]{\par\textbf{Задача: }#1}
\begin{document}
\begin{titlepage}
\begin{center}
\bfseries

{\Large Московский авиационный институт\\ (национальный исследовательский университет)

}

\vspace{48pt}

{\large Факультет прикладной математики и физики

}

\vspace{36pt}


{\large Кафедра вычислительной математики и~программирования

}


\vspace{48pt}

{ Лабораторная работа №2 по курсу <<Дискретный анализ>>}

\end{center}

\vspace{72pt}

\begin{flushright}
\begin{tabular}{rl}
Студент: & Д.\,Г. Кривенко \\
Преподаватель: & А.\,А. Кухтичев \\
Группа: & 80-206 \\
Дата: & \\
Оценка: & \\
Подпись: & \\
\end{tabular}
\end{flushright}

\vfill

\begin{center}
\bfseries
Москва, 2016
\end{center}
\end{titlepage}

\pagebreak
\newpage

\CWHeader{Лабораторная работа \textnumero 2}

\CWProblem{
Необходимо создать программную библиотеку, реализующую указанную структуру данных, на основе которой разработать программу -- словарь. В словаре каждому ключу, представляющему из себя регистронезависимую последовательность букв английского алфавита длиной не более 256 символов, поставлен в соответствие некоторый номер, от 0 до $2^{64} - 1$. Разным словам может быть поставлен в соответствие один и тот же номер.

{\bfseries Вариант структуры :} АВЛ -- дерево.
}
\pagebreak
\section{Описание}
Требуется реализовать АВЛ-дерево. \\ 
АВЛ--дерево -- сбалансированное по высоте двоичное дерево поиска: для каждой его вершины высота её двух поддеревьев различается не более чем на единицу\cite{wikipedia_tree}. Так как в ходе работы с деревом его высота может измениться, а баланс нарушиться, применяют такую оперцию как балансировка вершины. \\ 
Балансировкой вершины называется операция, которая в случае разницы высот левого и правого поддеревьев равной 2, изменяет связи предок -- потомок в поддереве данной вершины так, что разница становится меньше либо равной 1, иначе ничего не меняет\cite{wikipedia_tree}. Указанный результат получается вращениями поддерева данной вершины. \\
Существует несколько типов вращений: малое левое вращение, малое правое вращение, большое правое вращение, большое левое вращение. Большое вращение -- это комбинация правого и левого малых вращений.
Опишем псевдокодом малый левый поворот\cite{wikipedia_avl}:
\begin{lstlisting}[language=C++]
	function rotateLeft(Node a):
   Node b = a.right
   a.right = b.left
   b.left = a
   height adjustment
\end{lstlisting}
Опишем псевдокодом большой левый поворот\cite{wikipedia_avl}:
   \begin{lstlisting}[language=C++]
   function bigRotateLeft(Node a):
   rotateRight(a.right)
   rotateLeft(a)
\end{lstlisting}
Малое правое и большое правое вращение определяются симметрично малому левому и большому левому вращению.\\ 
Алгоритм добавления вершины\cite{wikipedia_tree}: 
\begin{enumerate}
	\item Проход по пути поиска, пока не убедимся, что ключа в дереве нет.
	\item Включение новой вершины в дерево.
	\item «Отступления» назад по пути поиска и проверка в каждой вершине показателя сбалансированности. Если необходимо — балансировка. 
\end{enumerate} 
Алгоритм удаления вершины:
\begin{enumerate}
	\item  Если вершина -- лист, то удалим её и вызовем балансировку всех её предков в порядке от родителя к корню.
	\item Найдём самую близкую по значению вершину в поддереве наибольшей высоты (правом или левом).
	\item Переместим найденную вершину на место удаляемой вершины.
	\item Вызовем процедуру удаления найденной вершины.
\end{enumerate} 
Операция поиска выполняется аналогично операции поиска в бинарном дереве.
\pagebreak
\section{Исходный код}
Программа должна обрабатывать строки входного файла до его окончания. Каждая строка может иметь следующий формат:

+ word 34 -- добавить слово «word» с номером 34 в словарь. Программа должна вывести строку «OK», если операция прошла успешно, «Exist», если слово уже находится в словаре.\\
- word -- удалить слово «word» из словаря. Программа должна вывести «OK», если слово существовало и было удалено, «NoSuchWord», если слово в словаре не было найдено.\\
word — найти в словаре слово «word». Программа должна вывести «OK: 34», если слово было найдено; число, которое следует за «OK:» -- номер, присвоенный слову при добавлении. В случае, если слово в словаре не было обнаружено, нужно вывести строку «NoSuchWord».\\
! Save /path/to/file -- сохранить словарь в бинарном компактном представлении на диск в файл, указанный параметром команды. В случае успеха, программа должна вывести «OK», в случае неудачи выполнения операции, программа должна вывести описание ошибки (см. ниже).\\
! Load /path/to/file -- загрузить словарь из файла. Предполагается, что файл был ранее подготовлен при помощи команды Save. В случае успеха, программа должна вывести строку «OК», а загруженный словарь должен заменить текущий (с которым происходит работа); в случае неуспеха, должна быть выведена диагностика, а рабочий словарь должен остаться без изменений. Кроме системных ошибок, программа должна корректно обрабатывать случаи несовпадения формата указанного файла и представления данных словаря во внешнем файле.\\
Создадим структуру $elem$ в которой будем хранить ключь и значение. Создадим структуру $node$ -- езел дерева, который содержит указатели на правого и левого потомка $left$ и $right$, показатель балансировки $bal$ и поле $data$ 
\begin{lstlisting}[language=C++]
struct elem{
    char* str;
    size_t value;
};
struct node{
    elem data;
    int bal;
    node* left;
    node* right;
    node(elem k) { data = k; left = right = NULL; bal = 0;}
    ~node(){};  
};	
\end{lstlisting}
Опишем функциии, необходимые для работы с деревом.
\begin{longtable}{|p{7.5cm}|p{7.5cm}|}
\hline
\rowcolor{lightgray}
\multicolumn{2}{|c|} {main.cpp}\\
\hline
node* RightRotation(node* root)& Малый правый поворот\\
\hline
node* LeftRotation(node* root)& Малый левый поворот\\
\hline
node* Balance(node* root)& Балансировка дерева\\
\hline
int  Add(node** rootp, elem val) & Добавление элемента в дерерво\\
\hline
node* Find(node** rootp, char val[])& Поиск элемента по значению\\
\hline
node* FindMin(node** rootp)& Нахождение минимального элемента\\
\hline
int Remove(node** rootp, char val[])& Удаление элемента\\
\hline
void Delete(node** rootp)& Удаление всего дерева\\
\hline
void Serialization(node** rootp, FILE* f) & Сериализация\\
\hline
node* Deserialisation(FILE* f) & Десериализация\\
\hline
\end{longtable}
\pagebreak

\section{Консоль}
\begin{alltt}
dmitriy@dmitriy-desktop:~$ g++ -pedantic -Wall -Werror -std=c++11 LR2.cpp
dmitriy@dmitriy-desktop:~$ ./a.out 
+ wtrttNwtwt 6356151351436705792
+ wtrttNwtw 11355346280444854272
+ wtrttNwt 6920329909665726464
+ wtrttNw 399671756802097152
+ wtrttN 11243149194066984960
+ wtrtt 5245798614692528128
+ wtrt 12751358511042461696
+ wtr 1814494297426624512
+ wt 9695799558577651712
+ w 3437433912018599936
OK
OK
OK
OK
OK
OK
OK
OK
OK
OK
w
OK: 3437433912018599936
- w
OK
w
NoSuchWord
+ wt 2342
Exist
! Save f1
OK
+ w 324
OK
w
OK: 324
! Load f2
ERROR: Couldn't create file
! Load f1
OK
w
NoSuchWord
\end{alltt}
\pagebreak
\section{Тест производительности}
Тест производительности представляет из себя следующее: реализованное АВЛ--дерево сравнивается со стандартным контейнером map. В ходе тестирования производятся запросы различного характера. Время измеряется в тактах процессора.\\
Выпоняются 100 запросов различного характера(удаление, добавление и тд).
\begin{alltt}
dmitriy@dmitriy-desktop:~$ g++ benchmark.cpp 
dmitriy@dmitriy-desktop:~$./a.out
AVL time: 1643 
MAP time: 944
\end{alltt}
Выполняютс $10^6$ запросов различного характера(удаление, добавление и тд).
\begin{alltt}
dmitriy@dmitriy-desktop:~$ ./a.out
AVL time: 7184635
MAP time: 20553412
\end{alltt}
Заметим, что реализованная структура и стандартный контейнер работают практически за одинаковое время.
\pagebreak
\section{Выводы}
Выполнив вторую лабораторную работу по курсу \enquote{Дискретный анализ}, я изучил АВЛ -- дерево, а так же научился его реализовывать. Выполняя данную работу я впервые столкнулся с операциями сериализации и десериализация. В процессе написания соответствующих функций я подробно изучил работу с бинарными файлами в C++: функции fopen, fwrite, fread и др. Так же разробрал и написал алгоритмы сериализации и десериализация для АВЛ -- дерева\pagebreak
\begin{thebibliography}{99}
\bibitem{wikipedia_tree}
{\itshape АВЛ -- дерево -- Википедия.} \\URL: \texttt{hhttps://ru.wikipedia.org/wiki/АВЛ--дерево} 
\bibitem{wikipedia_avl}
{\itshape АВЛ -- дерево -- Викиконспекты.} \\URL: \texttt{http://neerc.ifmo.ru/wiki/index.php?title=АВЛ--дерево} 
\end{thebibliography}
\pagebreak
\end {document}
